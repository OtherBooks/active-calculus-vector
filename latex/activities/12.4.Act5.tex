\begin{activity} \label{A:12.4.5}  
  Let $\vF=\langle F_1,F_2\rangle$ be a continuous, path-independent vector field on an open,
  connected region $D$. We will assume that $D$ is in $\R^2$ and $\vF$
  is a two-dimensional vector field, but the ideas below generalize
  completely to $\R^3$. We want to define a function $f$ on $D$ by
  using the vector field $\vF$ and line integrals, much like the
  Second Fundamental Theorem of Calculus allows us to define an
  antiderivative of a continuous function using a definite
  integral. To that end, we assign $f(x_0,y_0)$ an arbitrary
  value. (Setting $f(x_0,y_0)=0$ is probably convenient, but we won't
  explicitly tie our hands. Just assume that $f(x_0,y_0)$ is defined
  to be some number.) Now for any other point $(x,y)$ in $D$, define
  \[f(x,y) = f(x_0,y_0) + \int_C\vF\cdot d\vr,\]
  where $C$ is any oriented path from $(x_0,y_0)$ to $(x,y)$. Since
  $D$ is connected, such an oriented path must exist. Since $\vF$ is
  path-independent, $f$ is well-defined. If different paths from
  $(x_0,y_0)$ to $(x,y)$ gave
  different values for the line integral, then we'd not be sure what
  $f(x,y)$ really is.

  To better understand this mysterious function $f$ we've now defined,
  let's start looking at its partial derivatives.
  \ba
\item Since $D$ is open, there is a disc (perhaps very small)
  surrounding $(x,y)$ that is contained in $D$, so fix a point $(a,b)$
  in that disc. Since $D$ is connected, there is a path $C_1$ from
  $(x_0,y_0)$ to $(a,b)$. Let $C_y$ be the line segment from $(a,b)$
  to $(a,y)$ and let $C_x$ be the line segment from $(a,y)$ to
  $(x,y)$. (See Figure~\ref{F:12.4.paths-potential}.) Rewrite $f(x,y)$
  as a sum of $f(x_0,y_0)$ and line
  integrals along $C_1$, $C_y$, and $C_x$.
  \begin{figure}
    \centering
    \begin{overpic}{figures/fig_12_4_paths_potential.pdf}
      \put(10,0){$(x_0,y_0)$}
      \put(100,117){$(a,b)$}
      \put(65,160){$(a,y)$}
      \put(160,150){$(x,y)$}
      \put(75,140){$C_y$}
      \put(125,150){$C_x$}
      \put(30,70){$C_1$}

      \put(237,0){$(x_0,y_0)$}
      \put(310,110){$(a,b)$}
      \put(387,110){$(x,b)$}
      \put(370,160){$(x,y)$}
      \put(385,140){$L_y$}
      \put(353,110){$L_x$}
      \put(257,70){$C_1$}
    \end{overpic}
    \caption{Two piecewise smooth oriented curves from $(x_0,y_0)$ to
      $(x,y)$.}\label{F:12.4.paths-potential}
  \end{figure}
\item Notice that we can parameterize $C_y$ by $\langle a,t\rangle$
  for $b\leq t\leq y$. Find a similar parameterization for $C_x$.
\item Use the parameterizations from above to write
  $\int_{C_y}\vF\cdot d\vr$ and $\int_{C_x}\vF\cdot d\vr$ as single
  variable integrals in the manner of Section~\ref{S:12.3.ParamLineIntegrals}. (Recall
  that $\vF(x,y) = \langle F_1(x,y),F_2(x,y)\rangle$, enabling you to
  express your integrals in terms of $F_1$ and $F_2$ without any dot products.)
\item Rewrite your expression for $f(x,y)$ using the single variable
  integrals above (and a line integral along $C_1$). 
\item Use your expression for $f(x,y)$ and the Second Fundamental
  Theorem of Calculus to calculate $f_x(x,y)$.
\item To calculate $f_y(x,y)$, we continue to consider a path $C_1$
  from $(x_0,y_0)$ to $(a,b)$, but now let $L_x$ be the line segment
  from $(a,b)$ to $(x,b)$ and let $L_y$ be the line segment from
  $(x,b)$ to $(y,b)$. Modify the process you used to find $f_x(x,y)$
  to find $f_y(x,y)$.
\item What can you conclude about the relationship between $\nabla f$
  and $\vF$?
\ea
\end{activity}
\begin{smallhint}

\end{smallhint}
\begin{bighint}

\end{bighint}
\begin{activitySolution}

\end{activitySolution}
\aftera
%%% Local Variables:
%%% mode: latex
%%% TeX-master: "../0_AC_MV"
%%% End:
