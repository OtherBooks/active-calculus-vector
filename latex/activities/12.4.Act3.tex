\begin{activity} \label{A:12.4.3}  
Suppose that $\vF$ is a continuous path-independent vector field (in
$\R^2$ or $\R^3$) on some domain $D$. 
\ba
\item Let $P$ and $Q$ be points in $D$ and let $C_1$ and $C_2$ be
  oriented curves from $P$ to $Q$. What can you say about
  $\int_{C_1}\vF\cdot d\vr$ and   $\int_{C_2}\vF\cdot d\vr$?
\item Let $C = C_1 - C_2$. Explain why $C$ is a closed curve.
\item Calculate $\oint_C\vF\cdot d\vr$. (Recall that we sometimes use
  the symbol $\oint$ for a line integral when the curve is closed.)
\item What does the previous part show must be the value of   $\oint_C
  \vF\cdot d\vr$ for any closed curve $C$ 
and continuous path-independent vector field $\vF$?
\saveCount
\ea

From the first part of this activity, you now now that the line
integral around any closed curve in a path-independent vector field is
$0$. What can we say about the converse? That is, suppose that $\vF$
is a continuous vector field on a domain $D$ for which
$\oint_C\vF\cdot d\vr = 0$ for all closed curves $C$.
\ba
\restoreCount
\item Pick two points $P$ and $Q$ in $D$. Let $C_1$ and $C_2$ be
  oriented curves from $P$ to $Q$. What type of curve is $C = C_1 -
  C_2$?
\item What is $\oint_C\vF\cdot d\vr$? Why?
\item What does that tell you about the relationship between
  $\int_{C_1}\vF\cdot d\vr$ and $\int_{C_2}\vF\cdot d\vr$?
\item Explain why this shows that $\vF$ is path-independent.
\ea
\end{activity}
\begin{smallhint}

\end{smallhint}
\begin{bighint}

\end{bighint}
\begin{activitySolution}

\end{activitySolution}
\aftera
%%% Local Variables:
%%% mode: latex
%%% TeX-master: "../0_AC_MV"
%%% End:
