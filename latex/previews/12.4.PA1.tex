\begin{pa} \label{PA:12.4}

In Activity~\ref{A:12.3.2} we considered the vector field $\vF(x,y) =
\langle y^2,2xy+3\rangle$ and two different oriented curves from
$(1,0)$ to $(-1,0)$. We found that the value of the line integral of
$\vF$ was the same along those two oriented curves.
\ba
\item Verify that $\vF(x,y) = \langle y^2,2xy+3\rangle$ is a gradient
  vector field by showing that $\vF = \nabla f$ for the function
  $f(x,y) = xy^2 + 3y$.
\item Calculate $f(-1,0)-f(1,0)$. How does this value compare to the
  value of the line integral $\int_{C_1}\vF\cdot d\vr$ you found in
  Activity~\ref{A:12.3.2}?
\item Let $C_3$ be the line segment from $(1,1)$ to $(3,4)$. Calculate
  $\int_{C_3}\vF\cdot d\vr$ as well as $f(3,4)-f(1,1)$. What do you
  notice?
\saveCount
\ea

We've used Clairaut's Theorem to argue that a vector field in $\R^2$
is not a gradient vector field when $\partial F_1/\partial
y\neq \partial F_2/\partial x$, and earlier in this preview activity,
you verified that a given vector field was the gradient of a
particular function of two variables. Clairaut's Theorem holds for
functions of three variables. However, in that case there are six mixed partials to
calculate, and thus it can be rather tedious. The remaining parts of
this preview activity suggest a process for determining if a vector
field in $\R^3$ is conservative as well as finding a potential
function for the vector field.

Let $\vG(x,y,z) = \langle 3e^{y^2}+z\sin(x),6xy e^{y^2} -
  z,3z^2-y-\cos(x)\rangle$ and $\vH(x,y,z) = \langle 3x^2 y,x^3+2yz^3,xz+3y^2z^2\rangle$.
\ba
\restoreCount
\item If $\vG$ and $\vH$ are to be gradient vector fields, then there
  are functions $g$ and $h$ for which $\vG = \nabla g$ and $\vH=\nabla
  h$. What would this tell us about the partial derivatives $g_x$,
  $g_y$, $g_z$, $h_x$, $h_y$, and $h_z$?
\item Find a function $g$ so that $\partial g/\partial x =
  3e^{y^2}+z\sin(x)$. Find a function $h$ so that $\partial
  h/\partial x = 3x^2y$.
\item When finding the most general antiderivative for a function of
  one variable, we add a constant of integration (usually denoted by
  $C$) to capture the fact that any constant will vanish through
  differentiation. When taking the partial derivative with respect to
  $x$ of a function of
  $x$, $y$, and $z$, what variables can appear in terms that vanish
  because they are treated as constants? What does this tell you
  should be added to $g$ and $h$ in the previous part to make them
  the most general possible functions with the desired partial
  derivatives with respect to $x$?
\item Now calculate $\partial g/\partial y$ and $\partial
  h/\partial y$. Explain why this tells you that we must have
  \[g(x,y,z) = 3xe^{y^2}-z\cos(x)-yz+m_1(z)\]
  and
  \[h(x,y,z) = x^3y+y^2z^3+m_2(z)\]
  for some functions $m_1$ and $m_2$ depending only on $z$.
\item Calculate $\partial g/\partial z$ and $\partial h/\partial z$
  for the functions in the part above. Notice that $m_1$ and $m_2$ are
  functions of $z$ alone, so taking a partial derivative with respect
  to $z$ is the same as taking an ordinary derivative, and thus you
  may use the notation $m'_1(z)$ and $m'_2(z)$.
\item Explain why $\vG$ is a gradient vector field but $\vH$ is not a
  gradient vector field. Find a potential function for $\vG$.
\ea
\end{pa} 
\afterpa 
%%% Local Variables:
%%% mode: latex
%%% TeX-master: "../0_AC_MV"
%%% End:
