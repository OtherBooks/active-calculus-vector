\begin{pa} \label{PA:12.3}
Let $\vF=\langle xy,y^2\rangle$, let $C_1$ be the line segment from
from $(1,1)$ to $(4,1)$, let $C_2$ be the line segment from $(4,1)$ to
$(4,3)$, and let $C_3$ be the line segment from $(1,1)$ to
$(4,3)$. Also let $C = C_1 + C_2$. This vector field and the curves
are shown in Figure~\ref{F:12.3.field-segments}.
\begin{figure}
  \centering
  \begin{overpic}[width=0.55\linewidth]{fig_12_3_preview_field_segments.pdf}
    \put(130,60){$C_1$}
    \put(195,115){$C_2$}
    \put(120,143){$C_3$}
  \end{overpic}
  \caption{A vector field $\vF$ and three oriented curves.}
  \label{F:12.3.field-segments}
\end{figure}

\ba
\item Every point along $C_1$ has $y=1$. Therefore, along $C_1$, the
  vector field $\vF$ can be viewed purely as a function of $x$. In
  particular, along $C_1$, we have $\vF(x,1) = \langle
  x,1\rangle$.
  Since every point along $C_2$ has the same $x$-value, what (in terms
  of $y$ only) is $\vF$ along $C_2$?
\item Recall that $d\vr \approx \Delta \vr$, and along $C_1$, we have
  that $\Delta\vr = \Delta x\vi \approx dx\vi$. Thus, $d\vr = \langle
  dx,0\rangle$. We know that along $C_1$, $\vF = \langle
  x,1\rangle$. What does this mean $\vF\cdot d\vr$ is along $C_1$?
  What interval of $x$-values describes $C_1$? Use these facts
  fact to write $\int_{C_1} \vF\cdot d\vr$ as an integral of the form
  $\int_a^b f(x)\, dx$ and evaluate the integral. 
\item Use an analogous approach to write $\int_{C_2} \vF\cdot d\vr$ as
  a limit of the form $\int_\alpha^\beta g(y)\, dy$ and evaluate the
  integral.
\item Use the previous parts and a property of line integrals to
  calculate $\int_C\vF\cdot d\vr$ without having to evaluate any
  additional integrals. 
\item Evaluating $\int_{C_3}\vF\cdot d\vr$ takes more work at this
  stage, so let's break the process into smaller pieces.
  \begin{enumerate}[label=\roman*.]
\label{F:12.3.field-segments}  \item Since $C_3$ is a line segment, find the slope-intercept
    ($y=mx+b$) form of the equation of this line.
  \item Just as we noticed that along $C_1$ we always had $y=1$, we
    now know how to express $y$ in terms of $x$ for all points along
    $C_3$. Use this to to express $\vF(x,y) = \vF(x,mx+b)$ as a
    vector purely in terms of $x$ for points along $C_3$.
  \item We often think of the slope of a line as being $\Delta
    y/\Delta x$. Use this fact and the slope of the line containing
    $C_3$ to express $\Delta y$ as a multiple of $\Delta x$. 
  \item We may view $\Delta\vr$ as $\langle \Delta x,\Delta y\rangle$.
    Since $\Delta x\approx dx$ and $\Delta y\approx dy$,
    write $d\vr$ as a vector in terms of $dx$.
  \item Use the range of $x$-values covered by the line segment $C_3$
    to write $\int_{C_3}\vF\cdot d\vr$ as a single-variable integral
    of the form $\int_a^b f(x)\, dx$ and evaluate the integral.
  \end{enumerate}
\item Notice that $C$ and $C_3$ both start at $(1,1)$ and end at
  $(4,3)$. How do the values of $\int_C\vF\cdot d\vr$ and
  $\int_{C_3}\vF\cdot d\vr$ compare?
\item Is $\vF$ a gradient vector field? Why or why not?\emph{Hint}: If
  $\vF$ were a gradient vector field, then there would be a function
  $f$ such that $\vF = \nabla f$. What would Clairaut's theorem say in
  this case?
\ea
\end{pa} 
\afterpa 
%%% Local Variables:
%%% mode: latex
%%% TeX-master: "../0_AC_MV"
%%% End:
